\usepackage{amsthm, amsmath, amssymb}
\usepackage{bm}
\usepackage{booktabs}
\usepackage{mathrsfs}

\newcommand{\asim}{\stackrel{{}_{\bullet}}{\sim}}
\newcommand{\bA}{\bm A}
\newcommand{\bX}{\bm X}
\newcommand{\bY}{\bm Y}
\newcommand{\bmu}{\boldsymbol{\mu}}
\newcommand{\Bernoulli}{\operatorname{Bernoulli}}
\newcommand{\Beta}{\operatorname{Beta}}
\newcommand{\betat}{\widetilde\beta}
\newcommand{\Binomial}{\operatorname{Binomial}}
\newcommand{\Cov}{\operatorname{Cov}}
\newcommand{\Data}{\mathcal D}
\newcommand{\diag}{\operatorname{diag}}
\newcommand{\E}{\mathbb E}
\newcommand{\Fisher}{\mathcal I}
\newcommand{\Gam}{\operatorname{Gam}}
\newcommand{\gammat}{\widetilde\gamma}
\newcommand{\GP}{\operatorname{GP}}
\newcommand{\Identity}{\textrm{I}}
\newcommand{\iid}{\stackrel{\text{iid}}{\sim}}
\newcommand{\indep}{\stackrel{\text{ind}}{\sim}}
\newcommand{\logit}{\operatorname{logit}}
\newcommand{\MSE}{\operatorname{MSE}}
\newcommand{\Normal}{\operatorname{Normal}}
\newcommand{\Odds}{\operatorname{Odds}}
\newcommand{\OFisher}{\mathcal J}
\newcommand{\phit}{\widetilde \phi}
\newcommand{\Poisson}{\text{Poisson}}
\newcommand{\Reals}{\mathbb R}
\newcommand{\sH}{\mathcal H}
\newcommand{\sM}{\mathcal M}
\newcommand{\trace}{\operatorname{trace}}
\newcommand{\Tree}{\mathcal T}
\newcommand{\Uniform}{\operatorname{Uniform}}
\newcommand{\Var}{\operatorname{Var}}
\newcommand{\zeros}{\boldsymbol{0}}

% \newtheorem{examplefirst}{Example}
% \newtheorem{examplesecond}{Example}
% \newenvironment<>{examplefirst}[1][]{%
%   \setbeamercolor{block title example}{fg=white,bg=red!75!black}%
%   \begin{example}#2[#1]}{\end{example}}
% \newenvironment<>{examplesecond}[1][]{%
%   \setbeamercolor{block title example}{fg=white,bg=blue!75!black}%
%   \begin{example}#2[#1]}{\end{example}}
  
% \setbeamercolor{background canvas}{bg=white}
% \metroset{block=fill}

\usefonttheme{serif}
% \usetheme{Boadilla}
%gets rid of bottom navigation bars
\setbeamertemplate{footline}[frame number]{}

%gets rid of bottom navigation symbols
\setbeamertemplate{navigation symbols}{}

%gets rid of footer
%will override 'frame number' instruction above
%comment out to revert to previous/default definitions
% \setbeamertemplate{footline}{}
\usecolortheme{lily}
\setbeamercolor{block title}{bg=blue!20,fg=black}
\setbeamercolor{block body}{bg = blue!10, fg = black}
\setbeamertemplate{itemize item}[square]
\setbeamercolor{itemize item}{fg = cyan}
\setbeamercolor{itemize subitem}{fg = cyan}
\setbeamercolor{enumerate item}{fg = cyan}

\setbeamercolor{block title alerted}{fg=white, bg=orange}
% 2- Block body (background)
\setbeamercolor{block body alerted}{bg=orange!25}

\usetheme{default}
% \beamertemplatenavigationsymbolsempty

% \definecolor{fore}{RGB}{249,242,215}
% \definecolor{back}{RGB}{51,51,51}
% \definecolor{title}{RGB}{255,0,90}
% \definecolor{foo}{HTML}{E19898}


\setbeamercolor{titlelike}{fg=cyan}
% \setbeamercolor{normal text}{fg=fore,bg=back}

\definecolor{alertblue}{HTML}{00BDEC}
\newcommand{\alertb}[1]{{\color{alertblue}#1}}
\setbeamercolor{alerted text}{fg=orange}

\usepackage[most]{tcolorbox}
\usepackage{bm}

\definecolor{mygray}{gray}{0.9}

\newtcolorbox[auto counter]{myexample}[1][]{
  breakable,
  title=Example: #1,
  sharp corners,
  before upper=\setlength{\parskip}{5pt},
  borderline west={2pt}{0pt}{olive}, % straight vertical line at the left edge
  colback=white,
  boxrule=0pt, % no real frame,
  % colframe=lightgray,
  colbacktitle=lightgray,
  coltitle=black,
  fonttitle=\bfseries,
  enhanced jigsaw
}

\newtcolorbox[auto counter]{mytheorem}[1][]{
  breakable,
  title=Theorem: #1,
  sharp corners,
  before upper=\setlength{\parskip}{5pt},
  % borderline west={2pt}{0pt}{olive}, % straight vertical line at the left edge
  colback=mygray,
  boxrule=0pt, % no real frame,
  % colframe=lightgray,
  colbacktitle=mygray,
  coltitle=black,
  fonttitle=\bfseries,
  enhanced jigsaw
}

% \newtcolorbox[auto counter]{exercise}[1][]{%
%   breakable,
%   enhanced jigsaw, % better frame drawing
%   borderline west={2pt}{0pt}{red}, % straight vertical line at the left edge
%   sharp corners, % No rounded corners
%   boxrule=0pt, % no real frame,
%   fonttitle={\large\bfseries},
%   coltitle={black},  % Black colour for title
%   title={Exercise:\ },  % Fixed title
%   attach title to upper, % Move the title into the box
%   #1
% }

\newtcolorbox[auto counter]{myexercise}[1][]{
  breakable,
  title=Exercise: #1,
  sharp corners,
  before upper=\setlength{\parskip}{5pt},
  borderline west={2pt}{0pt}{cyan}, % straight vertical line at the left edge
  colback=white,
  boxrule=0pt, % no real frame,
  % colframe=lightgray,
  colbacktitle=lightgray,
  coltitle=black,
  fonttitle=\bfseries,
  enhanced jigsaw
  % frame hidden
}

\newtcolorbox[auto counter]{mydefinition}[1][]{
  breakable,
  title=Definition: #1,
  sharp corners,
  before upper=\setlength{\parskip}{5pt},
  borderline west={2pt}{0pt}{violet}, % straight vertical line at the left edge
  colback=white,
  boxrule=0pt, % no real frame,
  % colframe=lightgray,
  colbacktitle=lightgray,
  coltitle=black,
  fonttitle=\bfseries,
  enhanced jigsaw
}

\newtcolorbox{warning}[1][]{
  breakable,
  title=Warning!,
  sharp corners,
  colback=red!5!white, % Background color
  colframe=red!75!black, % Frame color
  colbacktitle=red!85!black, % Background color for the title
  coltitle=white, % Text color for the title
  fonttitle=\bfseries, % Bold title text
  #1 % For specifying additional options on the fly
}