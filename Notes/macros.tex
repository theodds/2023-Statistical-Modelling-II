\usepackage[most]{tcolorbox}
\usepackage{bm}

\newtcolorbox[auto counter]{myexample}[1][]{
  breakable,
  title=Example~\thetcbcounter: #1,
  sharp corners,
  before upper=\setlength{\parskip}{5pt},
  borderline west={2pt}{0pt}{olive}, % straight vertical line at the left edge
  colback=white,
  boxrule=0pt, % no real frame,
  % colframe=lightgray,
  colbacktitle=lightgray,
  coltitle=black,
  fonttitle=\bfseries,
  enhanced jigsaw
}

% \newtcolorbox[auto counter]{exercise}[1][]{%
%   breakable,
%   enhanced jigsaw, % better frame drawing
%   borderline west={2pt}{0pt}{red}, % straight vertical line at the left edge
%   sharp corners, % No rounded corners
%   boxrule=0pt, % no real frame,
%   fonttitle={\large\bfseries},
%   coltitle={black},  % Black colour for title
%   title={Exercise:\ },  % Fixed title
%   attach title to upper, % Move the title into the box
%   #1
% }

\newtcolorbox[auto counter]{myexercise}[1][]{
  breakable,
  title=Exercise~\thetcbcounter: #1,
  sharp corners,
  before upper=\setlength{\parskip}{5pt},
  borderline west={2pt}{0pt}{cyan}, % straight vertical line at the left edge
  colback=white,
  boxrule=0pt, % no real frame,
  % colframe=lightgray,
  colbacktitle=lightgray,
  coltitle=black,
  fonttitle=\bfseries,
  enhanced jigsaw
  % frame hidden
}

\newtcolorbox[auto counter]{definition}[1][]{
  breakable,
  title=Definition~\thetcbcounter: #1,
  sharp corners,
  before upper=\setlength{\parskip}{5pt},
  borderline west={2pt}{0pt}{violet}, % straight vertical line at the left edge
  colback=white,
  boxrule=0pt, % no real frame,
  % colframe=lightgray,
  colbacktitle=lightgray,
  coltitle=black,
  fonttitle=\bfseries,
  enhanced jigsaw
}

\makeatletter
\def\fps@figure{t}
\makeatother


\newcommand{\asim}{\stackrel{{}_{\bullet}}{\sim}}
\newcommand{\bX}{\mathbf{X}}
\newcommand{\bx}{\mathbf{x}}
\newcommand{\by}{\mathbf{y}}
\newcommand{\bY}{\mathbf{Y}}
\newcommand{\Bernoulli}{\operatorname{Bernoulli}}
\newcommand{\Binomial}{\operatorname{Binomial}}
\newcommand{\Cov}{\operatorname{Cov}}
\newcommand{\Data}{\mathcal D}
\newcommand{\diag}{\operatorname{diag}}
\newcommand{\Dirichlet}{\operatorname{Dirichlet}}
\newcommand{\E}{\mathbb E}
\newcommand{\Ell}{\mathscr L}
\newcommand{\EmpF}{\mathbb F}
\newcommand{\Fhat}{\widehat F}
\newcommand{\Fisher}{\mathcal I}
\newcommand{\GEM}{\operatorname{GEM}}
\newcommand{\Ghat}{\widehat G}
\newcommand{\Gam}{\operatorname{Gam}}
\newcommand{\GP}{\operatorname{GP}}
\newcommand{\Identity}{\mathrm{I}}
\newcommand{\iid}{\stackrel{\text{iid}}{\sim}}
\newcommand{\indep}{\stackrel{\text{indep}}{\sim}}
\newcommand{\logit}{\operatorname{logit}}
\newcommand{\MSE}{\operatorname{MSE}}
\newcommand{\muhat}{\widehat\mu}
\newcommand{\Odds}{\operatorname{Odds}}
\newcommand{\OFisher}{\mathcal J}
\newcommand{\psihat}{\widehat\psi}
\newcommand{\Normal}{\operatorname{Normal}}
\newcommand{\Poisson}{\operatorname{Poisson}}
\newcommand{\Reals}{\mathbb R}
\newcommand{\sC}{\mathcal C}
\newcommand{\sF}{\mathscr F}
\newcommand{\sG}{\mathcal G}
\newcommand{\Var}{\operatorname{Var}}
\newcommand{\zeros}{\boldsymbol{0}}
\newcommand{\bmu}{\boldsymbol \mu}