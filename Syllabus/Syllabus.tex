\documentclass[12pt]{article}
\textwidth=7in
\textheight=9.5in
\topmargin=-1in
\headheight=0in
\headsep=.5in
\hoffset= -.85in
\usepackage{termcal}
\usepackage{amsmath}
\usepackage{color}
\usepackage{comment}

\usepackage{parskip}

\renewcommand{\thefootnote}{\fnsymbol{footnote}}
\definecolor{tonyorange}{RGB}{217, 109, 0}
\newcommand{\alert}[1]{{\color{tonyorange} #1}}
\newenvironment{syllasec}[1]{\vskip.25in \noindent {\bf #1:}}{}
\newenvironment{dotslist}{\begin{center} \begin{minipage}{5in} 
      \begin{flushleft}}{\end{flushleft}\end{minipage}\end{center}}
% Few useful commands (our classes always meet either on Monday and Wednesday 
% or on Tuesday and Thursday)


\newcommand{\MWClass}{%
\calday[Monday]{\classday} % Monday
\skipday % Tuesday (no class)
\calday[Wednesday]{\classday} % Wednesday
\skipday % Thursday (no class)
\skipday % Friday 
\skipday\skipday % weekend (no class)
}

\newcommand{\TRClass}{%
\skipday % Monday (no class)
\calday[Tuesday]{\classday} % Tuesday
\skipday % Wednesday (no class)
\calday[Thursday]{\classday} % Thursday
\skipday % Friday 
\skipday\skipday % weekend (no class)
}

\newcommand{\Holiday}[2]{%
\options{#1}{\noclassday}
\caltext{#1}{#2}
}

\newcommand{\foo}{\alert{foo}}

\RequirePackage[OT1]{fontenc}
\RequirePackage[colorlinks,linkcolor=blue,citecolor=blue,urlcolor=blue]{hyperref}

\begin{document}
\newcommand{\TBA}{{\color{tonyorange} TBA}}


\begin{center}
{\bf SDS 383D: Statistical Modeling II \\
   MW 3:00PM  --- 4:30PM, Room: Zoom
}
\end{center}

\setlength{\unitlength}{1in}

\begin{picture}(6,.1) 
\put(0,0) {\line(1,0){6.25}}         
\end{picture}


\renewcommand{\arraystretch}{2}

\begin{syllasec}{Instructor}
  Dr. Antonio Linero, WEL5.244

  \noindent {\bf Office Hours:} 4:30pm --- 5:30PM MW, or by appointment.

  \noindent {\bf Contact:} \href{mailto:antonio.linero@austin.utexas.edu}{antonio.linero@austin.utexas.edu}
\end{syllasec}


\begin{syllasec}{Prerequisites}
  This course primarily concerns probabilistic modeling using generalized linear
  models and hierarchical Bayesian modeling, although there will be other topics
  covered. This course is intended for graduate students in the Statistics and
  Data Sciences Department or any students from other departments who are
  interested in developing/using probabilistic models. In terms of substantive
  prerequisites, I assume that you are comfortable with the following topics:
  \begin{itemize}
  \item linear algebra;
  \item how to program in a language like \texttt{R} or \texttt{Python};
  \item multivariate calculus;
  \item probability (although measure theory isn't necessary);
  \item basic inferential statistics at the level of Casella \& Berger;
  \item linear regression.
  \end{itemize}
  If you have any doubt about your preparation for this course, or have any
  questions about whether this course is right for you, feel free to chat with
  me.
\end{syllasec}

\begin{syllasec}{Course Website}
  Notes for this course are available at:
  \begin{center}
    \url{https://github.com/theodds/2023-Statistical-Modelling-II}
  \end{center}
  Syllabus and zoom meeting link for office hours are available at the course
  site on \url{canvas.utexas.edu}.
\end{syllasec}

\begin{syllasec}{Course Structure}
  This course is a blend between a traditional lecture-based course and a
  flipped classroom. Some of the time is spent on lectures in class. But a lot
  of the other class time will be student led. You will work on the exercises
  assigned on the course website and when you come to class you will share what
  you have done, and benefit from understanding what others have done. We will
  end up covering less than in an exclusively lecture-based course. But what you
  learn, you will learn deeply.
\end{syllasec}

\begin{syllasec}{Textbook} 
  Relevant readings will be given throughout the course. There are no formally
  required textbooks, but here are three recommended references that should be
  easily founder online:
  \begin{itemize}
  \item Data Analysis Using Regression and Multilevel/Hierarchical Models by
    Gelman and Hill. An e-book version is available through the UT Library
    website.
  \item Generalized Linear Models by McCullagh and Nelder.
  \item All of Statistics by Larry Wasserman.
  \end{itemize}
\end{syllasec}

\begin{syllasec}{Software}
  The examples in this course will use the \texttt{R} programming language,
  available at \url{www.r-project.org}. See also \url{www.rstudio.com} for a
nice development environment. You will be asked to submit \texttt{R}-code as
part of your homework solutions.
\end{syllasec}

\begin{syllasec}{Course Objectives}
  The purpose of this course is to give students the mathematical tools and
  intuition required to develop and deploy sophisticated probabilistic models.
  Major topics will include:
  \begin{itemize}
  \item classical Frequentist and Bayesian approaches to inference;
  \item fundamentals of generalized linear models;
  \item multivariate distributions, such as the multivariate Gaussian and
    Dirichlet;
  \item nonparametric regression techniques (smoothing, additive models, and
    Gaussian processes);
  \item hierarchical linear/generalized linear models.
  \end{itemize}
\end{syllasec}

\begin{syllasec}{Grading}
  Your grade consists of three pieces: 40\% exercises, 40\% final project, and
  20\% participation.

  \textbf{Homework will be due on the first class day of each month.} To lighten
  the burden, we will divide into groups of three each month (groups will not be
  allowed to overlap in subsequent months). \textbf{Groups will be randomly
    selected to present exercises for various problems, with supporting code
    included, throughout the semester.} At the end of each lecture, I will give
  a list of homework problems that I may ask you to present a solution to during
  the next lecture, but it is expected that you and your group will do all of
  the exercises in the notes.

  I will grade homework submissions in the following way: each month I will
  select ten problems at random to grade. All parts of all problems will be
  weighted equally so that, for example, if I graded Problem 1, Problem 2a, and
  Problem 2b, then each of these would account for 1/3rd of the grade for the
  month. Each part will be graded either 0/2, 1/2, or 2/2.

  As far as how submissions should look:
  \begin{enumerate}
  \item For mathematics questions, I'm fine with you turning in hand-written
    solutions, provided that you answer the questions using complete sentences.
    Answers should not consist of just long strings of calculations with no
    comments; you need to explain what you are doing.
  \item For data analysis or coding questions, I expect results to be typed up
    with code given in-line and explained. It should be structured more-or-less
    like how I have things written up in the notes (i.e., code and discussion
    given together). The easiest way to do this would be to submit an R Markdown
    or Quarto document, or create a Jupyter notebook. If using one of these
    frameworks, you should submit both the compiled document as a pdf as well as
    the raw notebook/.rmd/.qmd document.
  \end{enumerate}

  For each homework submission, you will give a grade to each of your group
  members (including yourself) that reflects the grade that you would assign
  that group member based on their contribution to the homework assignments and
  in-class presentations, as well as any notes that you have about how the month
  went. \textbf{This, in addition to participation in class, will be used to
    assign participation credit.} This, in combination with the homework turned
  in, will be used to assign a grade for the homework each month. \emph{Note:} I
  will still have final say on the grade each student gets on homework, and may
  overrule any good/bad evaluations if I feel (and have evidence supporting)
  that the evaluations are not representative of a student's contributions.
\end{syllasec}

\begin{syllasec}{Final Project}
  Pick some relevant topic that interests you. Clear it with me ahead of time, and
  then do it. Basically, I trust you to choose something that will optimize your
  own learning experience, and that will dovetail with your research and
  educational goals. It certainly can overlap with your own research. Examples:
  \begin{enumerate}
  \item Analyze a data set from your own research, using techniques from class
    or closely related techniques.
  \item Invent a new technique and show how awesome it is.
  \item Prove something interesting about a procedure or algorithm related to
    what we're studying (admittedly unlikely, but certainly possible!).
  \item Read a paper, or a group of related papers, that expands on some topic
    we've covered in class. Implement the method(s) and benchmark it (them)
    against something else.
  \end{enumerate}
  Final projects are due on the day of the university-scheduled exam: Friday,
  May 4th, 2024. \textbf{Note: you should feel free to work either solo or in
    pairs for the projects.}

  
  You should prepare a report up to 8 pages in length using the NeurIPS style
  files and guidelines, available on the course website. Reports are due on May
  4th.
\end{syllasec}

\begin{syllasec}{Students with Disabilities}
  Students with disabilities may request appropriate academic accommodations
  from the Division of Diversity and Community Engagement, Services for Students
  with Disabilities, 512-471-6259,
  \texttt{http://www.utexas.edu/diversity/ddce/ssd/}.
\end{syllasec}

\begin{syllasec}{Religious Holy Days}
  By UT Austin policy, you must notify me of your pending absence at least
  fourteen days prior to the date of observance of a religious holy day. If you
  must miss a class, an examination, a work assignment, or a project in order to
  observe a religious holy day, you will be given an opportunity to complete the
  missed work within a reasonable time after the absence.
\end{syllasec}

\begin{syllasec}{Scholastic Honesty}
  We expect students to behave with integrity. Students found cheating on exams
  or homework will receive a score of zero for that exam or assignment, and may
  be subject to additional disciplinary action. For more information on the
  University of Texas scholastic dishonesty policy, see the 2006-2007 General
  Information Catalog, Appendix C.
\end{syllasec}

\begin{syllasec}{Campus Safety}
  Please note the following recommendations regarding emergency evacuation from
  the Office of Campus Safety and Security, 512-471-5767,
  \texttt{http://www.utexas.edu/safety}:
  \begin{itemize}
  \item Occupants of buildings on The University of Texas at Austin campus are
    required to evacuate buildings when a fire alarm is activated. Alarm
    activation or announcement requires exiting and assembling outside.
  \item Familiarize yourself with all exit doors of each classroom and building
    you may occupy. Remember that the nearest exit door may not be the one you
    used when entering the building.
  \item Students requiring assistance in evacuation should inform the instructor
    in writing during the first week of class.
  \item In the event of an evacuation, follow the instruction of faculty or
    class instructors.
  \item Do not re-enter a building unless given instructions by the following:
    Austin Fire Department, The University of Texas at Austin Police Department,
    or Fire Prevention Services office.
  \item Behavior Concerns Advice Line (BCAL): 512-232-5050
  \item Further information regarding emergency evacuation routes and emergency
    procedures can be found at: \texttt{http://www.utexas.edu/emergency}.
  \end{itemize}
\end{syllasec}



\end{document}


%  LocalWords:  tonyorange OSB Gelman Dunson Vehtari factorizations variational
%  LocalWords:  autoencoders TBA SDS MW CMA Antonio Linero GDC pm Casella Rubin
% LocalWords:  website Carlin conjugacy Frequentist Dirichlet nonparametric UT
% LocalWords:  Monte Carlo GitHub NeurIPS BCAL
